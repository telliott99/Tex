\documentclass[11pt, oneside]{article} 
\usepackage{geometry}
\geometry{letterpaper} 
\usepackage{graphicx}
	
\usepackage{amssymb}
\usepackage{amsmath}
\usepackage{parskip}
\usepackage{color}
\usepackage{hyperref}

\graphicspath{{/Users/telliott_admin/Dropbox/Tex/png/}}
% \begin{center} \includegraphics [scale=0.4] {gauss3.png} \end{center}

\title{Archimedean property}
\date{}

\begin{document}
\maketitle
\Large

This property can be stated in a variety of forms.  One statement is that the real numbers are not bounded above in $\mathbb{N}$.  No matter how large a real number $x$ that we take, we can always find an integer that is larger.

Statements of the theorem all start with this:  for any real number $x$ ($\forall x \in \mathbb{R}$), we can find an integer $n$ such that ($\exists \ n \in \mathbb{N} \ | \ \ $):

$\bullet$  $n > x$

Or, for any real $a$ \emph{however small} we can find

$na > x$.

In the immortal words of xxx:  if we have a bathtub full of water and a teaspoon, we can empty the bathtub (given enough time).

If you prefer a small real number like $\epsilon$ ($\forall \epsilon \in \mathbb{R}$)

$\bullet$  $\frac{1}{n} < \epsilon$

Beck says:

\begin{quote}Theorem 7.6 (the Archimedean property) essentially says that \textbf{infinity is not part of the real numbers}... The Archimedean Property underlies the construction of an infinite decimal expansion for any real number, while the Monotone Sequence Property shows that any such infinite decimal expansion actually converges to a real number.\end{quote}

Apostol goes through this development:

$\bullet$  The set $\mathbf{P}$ of positive integers is \emph{unbounded above}.  The proof is to assume that $P$ is bounded above.  Then there is a largest element $n$ of $\mathbf{P}$ which is less than the bound.  

But by definition $n + 1$ is $\in \mathbf{P}$.

$\bullet$  For every real $x$ there exists a positive integer $n$ such that $n > x$.  Proof:  if this were not so, then $x$ would be an upper bound for $\mathbf{P}$.

Now, simply replace $x$ with $y/x$:

$\bullet$  For every real $y/x$ there exists a positive integer $n$ such that $n > y/x$.  Thus $nx > y$.

Apostol: 

\begin{quote}Geometrically it means that any line segment, no matter how long, may be covered by a finite number of line segments of a given positive length, no matter how small. In other words, a small ruler used often enough can measure arbitrarily large distances. Archimedes realized that this was a fundamental property of the straight line and stated it explicitly as one of the axioms of geometry.\end{quote}

Stewart's definition is:

Given a real number $\epsilon > 0$, there exists a positive integer $n$ such that
\[ \frac{1}{10^n} < \epsilon \]

This is certainly compatible with the other definitions.  If $n$ is an integer than so is $10^n$.  So $\epsilon$ is Apostol's (small) positive length and if we can choose $N$ so that $N \epsilon$ is as large as we please, we can certainly choose it so that $N \epsilon > 1$.

I interpret this as follows:  in distinguishing two real numbers $a$ and $b$ (really, trying to find another number in the gap between them), if $a - b = \epsilon$ is the distance between them, we can always find 
\[ \frac{1}{10^n} < \epsilon \]
and so always find another number (either real or rational) that lies between $a$ and $b$.

\end{document}}