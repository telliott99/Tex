\documentclass[11pt, oneside]{article}   	% use "amsart" instead of "article" for AMSLaTeX format
\usepackage{geometry}                		% See geometry.pdf to learn the layout options. There are lots.
\geometry{letterpaper}                   		% ... or a4paper or a5paper or ... 
%\geometry{landscape}                		% Activate for for rotated page geometry
%\usepackage[parfill]{parskip}    		% Activate to begin paragraphs with an empty line rather than an indent
\usepackage{graphicx}				% Use pdf, png, jpg, or eps§ with pdflatex; use eps in DVI mode
								% TeX will automatically convert eps --> pdf in pdflatex		
\usepackage{amssymb}
\usepackage{amsmath}
\usepackage{parskip}
\usepackage{color}
\usepackage{hyperref}

\title{Inductive proof of the binomial formula}
%\author{The Author}
\date{}							% Activate to display a given date or no date

\graphicspath{{/Users/telliott_admin/Dropbox/Tex/png/}}
\usepackage{hyperref}

\begin{document}

\maketitle
%\section{}
%\subsection{}
\Large
A concise statement of the binomial formula is that the general term is of the form
\[ {{n}\choose{k}} \ a^{n-k} \ b^k  \]
and the whole sum is
\[ \sum_{k=0}^{n} {{n}\choose{k}} \ a^{n-k} \ b^k  \]

where (from the theory of combinations):
\[  {{n}\choose{k}} = \frac{n!}{(n-k)! \ k!} \]
To do an actual calculation, we would first cancel the factor of $(n-k)!$ on top and bottom yielding
\[  {{n}\choose{k}} = \frac{n \times (n-1) \dots \times (n-k+1)}{k!} \]

\subsection*{Pascal's Lemma}
In order to prove the theorem (using induction) we will need the following result:

\[  {{n}\choose{k}} = {{n-1}\choose{k}} + {{n-1}\choose{k-1}} \]

Here is a simple proof from the theory of combinations.  Imagine that we are considering how many ways there are of forming a committee of $k$ members from a total of $n$ people.

We know that the number of ways of doing this is of course just
\[  {{n}\choose{k}} \]

Now, suppose that among these $n$ people we focus on one particular person, call her Alice.  Then there are two types of committees in our collection of combinations:  those in which Alice is a member, and those in which she is not.

For all committees of the first type, in addition to Alice, the other $k-1$ members must be drawn from $n-1$ people. The number of ways of doing this is
\[  {{n-1}\choose{k-1}} \]

For the second case, where Alice is not a member, we must recruit all $k$ members from $n-1$ people, since we are leaving Alice out.  The number of ways of doing this is
\[  {{n-1}\choose{k}} \]

But putting them together, these must be equal to the total number obtained by the standard analysis, and hence we have that
\[  {{n}\choose{k}} = {{n-1}\choose{k}} + {{n-1}\choose{k-1}} \]

This is effectively what we said near the beginning of the introduction to the binomial theorem:  in computing the $k$th coefficient in the $n$th row, we add the $k$ and $k-1$ values from the preceding row.

This preliminary result is called Pascal's Lemma and it is really the heart of the proof.

In using it below, we will alter its form slightly by substituting $n+1$ for $n$.  Thus:
\[  {{n+1}\choose{k}} = {{n}\choose{k}} + {{n}\choose{k-1}} \]

\subsection*{inductive proof}

If we look at one more term in the expansion for $(a+b)^n$, writing the term preceding the one given above, we have
\[ (a+b)^n = \dots + {{n}\choose{k-1}} \ a^{n-k+1} \ b^{k-1} + {{n}\choose{k}} \ a^{n-k} \ b^k + \dots  \]
Notice that the exponent increases by one for $a$ as we move to the left, while it decreases by one for $b$.

When we form the new general term in the expansion for $(a+b)^{n+1}$, as we said before, we multiply the first term by $b$ and the second one by $a$, obtaining
\[ \dots + b {{n}\choose{k-1}} \ a^{n-k+1} \ b^{k-1} + a {{n}\choose{k}} \ a^{n-k} \ b^k + \dots  \]
\[ = \dots +  {{n}\choose{k-1}} \ a^{n-k+1} \ b^{k} + {{n}\choose{k}} \ a^{n-k + 1} \ b^k + \dots  \]
But these two powers are the same, and since
\[ n-k+1 = (n + 1) - k \]
their sum is the general term in the expansion for $(a+b)^{n+1}$
\[ = \dots + C \ a^{(n+1)-k} \ b^k + \dots  \]


In adding them, we add their coefficients:
\[ (a+b)^{n+1} = \dots +  \ [ \ {{n}\choose{k-1}} + {{n}\choose{k}} \ ] \  a^{(n+1)-k} \ b^{k} + \dots  \]

Referring back to Pascal's Lemma, we substitute that result
\[  {{n+1}\choose{k}} = {{n}\choose{k}} + {{n}\choose{k-1}} \]
yielding
\[ (a+b)^{n+1} = \dots +  \ {{n+1}\choose{k}} + a^{(n+1)-k} \ b^{k} + \dots  \]

We obtain the general formula for the $n+1$ expansion.  This completes the inductive part of the proof.  It remains to check the binomial formula for a base case like $n=1$ or $n=2$, which I invite you to do.
$\blacksquare$


\end{document}  