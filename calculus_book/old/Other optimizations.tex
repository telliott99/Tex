\documentclass[11pt, oneside]{article} 
\usepackage{geometry}
\geometry{letterpaper} 
\usepackage{graphicx}
	
\usepackage{amssymb}
\usepackage{amsmath}
\usepackage{parskip}
\usepackage{color}
\usepackage{hyperref}

\graphicspath{{/Users/telliott_admin/Tex/png/}}
% \begin{center} \includegraphics [scale=0.4] {gauss3.png} \end{center}

%break
\title{Optimization problems}
\date{}

\begin{document}
\maketitle
\Large

\subsection*{Projectile range}

Suppose we fire a cannon where the ball has velocity $v$ at an angle $\theta$ with the horizon (straight up would be $\pi/2$ radians).  We wish to determine the angle that will give the maximum range.  We did this problem in a physics section, but it won't hurt to look at it again here.

The problem has a trick, namely that the distance in the horizontal or $x$-direction depends on the time (and therefore distance) in the $y$-direction, since when $y=0$, the cannonball will fall to earth and not move any more.

If you draw a diagram you will see that 
\[ v_y = v \sin \theta \]
where $v_y$ is the initial velocity in the $y$-direction.

The basic equation of motion under gravity is that 
\[ y = v_y t - \frac{1}{2}gt^2 \]
with $g=32$ so
\[ y = v_y t - 16t^2 \]

At the point of interest $y=0$ so
\[ 0 = v_y t - 16t^2 \]
\[ v_y t = 16 t^2 \]

This has two solutions, namely $t=0$ (not what we are interested in) and
\[ v_y = 16 t \]
\[ t = \frac{v_y}{16} \]
\[ = \frac{v \sin \theta}{16} \]

On the other hand, the quantity we are really interested in is the distance in the $x$-direction.  Similarly to $v_y$, 

\[ v_x = v \cos \theta \]
\[ x = v_x t = v \cos \theta \ \frac{v \sin \theta}{16} \]
\[ = \frac{v^2}{16} \ \cos \theta \sin \theta \]

This looks a little strange but all it really says is that the range is a function of the angle $\theta$ (and also of the square of the velocity).  If you do dimensional analysis at this point you might also be confused unless you remember that the factor of $16$ has units of meters per second squared.

We take the first derivative and set it equal to zero:

\[ x' = \frac{v^2}{16} \ (\cos^2 \theta - \sin^2 \theta ) = 0 \]

$v = 0$ gives a solution to this equation, but it's not the solution we want.  We want the solution given by

\[ \cos^2 \theta - \sin^2 \theta = 0 \]

The velocity and the gravitational constant have dropped out, which makes sense.  It makes intuitive sense that the angle for maximum range (given a velocity), should not depend on that velocity.

The above expression is zero when $\cos \theta = \sin \theta$.  If you don't see this you can say:

\[ 1 - \sin^2 \theta - \sin^2 \theta = 0 \]
\[ \sin^2 \theta = \frac{1}{2} \]
\[ \sin \theta = \frac{1}{\sqrt{2}} \]
\[ \theta = \frac{\pi}{4} \]

An elevation of $45$ degrees gives the maximum range.

\subsection*{Closest point to a parabola}

Suppose we consider the simple parabola
\[ y = x^2 \]

Our problem is to find the point(s) $(x,y)$ on the parabola that have the shortest distance to $P=(0,1)$.

One possibility is that $(0,0)$ is the minimum.  But it will turn out that it is not, and so there will be two such points, which are symmetrical about the $y-$axis.  Therefore, we consider only $x \ge 0$.

The distance from any point $(x,y)$ to $P=(0,1)$ is
\[ d = \sqrt{(0-x)^2 + (1-y)^2} \]

It is the case that if we minimize $d^2$, we also minimize $d$, so let's rewrite the equation as

\[ D = (0-x)^2 + (1-y)^2 \]
\[ D = x^2 + 1 - 2y + y^2 \]

Now, the constraint is that $y=x^2$ so plugging in we get

\[ D = y + 1 - 2y + y^2 \]
\[ = 1 - y + y^2 \]

Take the first derivative (with respect to $y$) and set it equal to zero:

\[ D' = -1 + 2y = 0 \]
\[ y = \frac{1}{2} \]
\[ x = \frac{1}{\sqrt{2}} \]

Check the actual distance:

\[ d = \sqrt{(0-x)^2 + (1-y)^2} \]
\[ = \sqrt{(\frac{1}{\sqrt{2}})^2 + (1- \frac{1}{2})^2 } \]
\[ = \sqrt{\frac{1}{2} + \frac{1}{4}} \]
\[ = \frac{\sqrt{3}}{2} \]
\[ = 0.866 \]

\[ (x,y) = (\frac{1}{\sqrt{2}}, \frac{1}{2}) \]

Note that $(1/\sqrt{2},1/2)$ is closer to $(0,1)$ than is $(0,0)$, as we said.

The slope of the line from $(0,1)$ to our point $(1/\sqrt{2},1/2)$ is
\[ \frac{\Delta y}{\Delta x} = \frac{1-1/2}{1/\sqrt{2}} = -\frac{1}{\sqrt{2}} \]

The slope of the tangent to the parabola is $2x$, and at $(1/\sqrt{2},1/2)$ it is

\[ m = 2x = 2 \frac{1}{\sqrt{2}} = \sqrt{2} \]

Since the product of the slopes is $-1$, the line corresponding to the minimum distance is perpendicular to the tangent.

The remainder of the problems are about maximizing angles.  The first two are variants with a slight twise, and the third 

\subsection*{movie screen}
A movie screen on a wall is 20 feet high and 10 feet above the floor. At what distance x from the front of the room should you position yourself so that the viewing angle $ \theta $ of the movie screen is as large as possible ?
\begin{center} \includegraphics [scale=0.4] {movie_screen.png} \end{center}

Somehow we need to maximize $\theta$ as a function of $x$, but we are given values that form the tangent of two angles.

However, we know that for $\theta$ in the half-open interval $[0, \pi/2)$, as $\theta$ increases so does $\tan \theta$.  Therefore, if we maximize $\tan \theta$, then $\theta$ will also be a maximum.

Let's use $s$ for the entire angle and $t$ for the lower triangle, then
\[ \theta = s - t \]
and the values we were given correspond to the tangents of $s$ and $t$.

We derived $\tan s - t$ before, and can do it again:
\[ \tan s - t = \frac{\sin s - t}{\cos s - t} \]
\[ = \frac{\sin s \cos t - \cos s \sin t}{\cos s \cos t + \sin s \sin t} \]
\[ = \frac{\tan s - \tan t}{1 + \tan s \tan t} \]

Plugging in the values provided:
\[ \tan \theta = \tan s - t \]
\[ = \frac{30/x - 10/x}{1 + 300/x^2} \]
\[ = 20 \ \frac{x}{x^2 + 300} \]
We take the derivative and set it equal to zero.  We can ignore the leading factor of $20$, obtaining
\[ 0 = \frac{x^2 + 300 - 2x^2}{(x^2 + 300)^2} = \frac{-x^2 + 300}{(x^2 + 300)^2} \]

This is equal to zero when the numerator is zero, that is, when
\[ x = \pm \sqrt{300} \]
Since $x$ is a distance we take the positive square root.
\[ x = \sqrt{300} = 10 \sqrt{3} \]

The angles are worth working out.  The tangent of the lower angle $t$ is $1/\sqrt{3}$.  This is a right triangle with hypotenuse equal to $2$ and sine equal to $1/2$.  Therefore $t = \pi/6$.

The tangent of the entire angle $s$ is equal to $3/\sqrt{3} = \sqrt{3}$.  This is a right triangle with hypotenuse equal to $2$ and cosine equal to $1/2$.  Therefore $s = \pi/3$.  

Therefore the angle to the screen $\theta = s - t$ at the maximum is $\pi/6$ or 30 degrees.

\subsection*{movie screen variant}
Stewart has a variant of this problem with a more complicated setup.
\begin{center} \includegraphics [scale=0.4] {movie_screen2.png} \end{center}

\begin{center} \includegraphics [scale=0.4] {movie_screen_q.png} \end{center}
It took me a while to even understand this version.  We ignore the complication of discrete rows, for the moment.  Our plan is to find an expression for the lengths of the dashed yellow lines, and use the law of cosines to find $\theta$ in terms of $x$.  Then we'll maximize $\theta$.

The horizontal position of your eyes is
\[ x \cos \alpha + 9 \]
The vertical position is
\[ x \sin \alpha + 4 \]
The vertical distance to the bottom of the screen from your eyes is:
\[ |(x \sin \alpha + 4) - 10| = |x \sin \alpha - 6| \]
The squared length of the lower dashed line is
\[ b^2 = (x \cos \alpha + 9)^2 + (x \sin \alpha - 6)^2 \]

The vertical distance to the top of the screen from your eyes is (since we know we won't be above the top of the screen):
\[ 35 - (x \sin \alpha + 4) = - x \sin \alpha + 31 \]
The squared length of the upper dashed line is
\[ a^2 = (x \cos \alpha + 9)^2 + (- x \sin \alpha + 31)^2 \]

The law of cosines is then
\[ c^2 = a^2 + b^2 - 2 a b \cos C \]
\[ 625 = a^2 + b^2 - 2 a b \cos \theta \]
\[ \cos \theta = \frac{a^2 + b^2 - 625}{2ab} \]

This matches his answer, part 1.  To maximize $\theta$ minimize $\cos \theta$.  The actual maximization is to be carried out on a fancy calculator.

I thought I could do this with the sum of tangents formula but it was a big mess.  So I wrote this script:

\begin{verbatim}
from math import *

c = 0.93969 # cos 20 degrees
s = 0.34202 # sin 20 degrees

def f(x):
    hoz = 9 + x*c
    a_sq = hoz**2 + (35 - 4 - x*s)**2
    b_sq = hoz**2 + (4 + x*s - 10)**2
    
    a = sqrt(a_sq)
    b = sqrt(b_sq)
    num = a_sq + b_sq - 25**2
    den = 2 * a * b
    return num*1.0/den

for x in range(1,90):
    print str(x).rjust(2),
    cosine = f(x)
    print acos(cosine) * 360 /(2 * pi)
\end{verbatim}

which shows a maximum $\theta = 48.52$ degrees for $x = 8$.  The seats are set at discrete intervals, the closest to the maximum is the third row, at $x = 9$.

For $x = 9$ and $\sin$ 20 degrees = $0.342$, the vertical position of your eyes is about $4 + 3.1 = 7.1$ feet, nearly $3$ feet below the bottom of the screen.  

We have
\[ \tan s - t = \frac{\sin s \cos t - \sin t \cos s}{\cos s \cos t  + \sin s \sin t} \]
\[ = \frac{\cos s(\sin s - \sin t)}{\cos^2 s + \sin s \sin t} \]

since we know that the position is less than $10$ we can write
\[ \sin s = 31 - x \sin \alpha \]
\[ \sin t = 6 - x \sin \alpha \]
\[ \cos s = 4 - x \cos \alpha \]
Therefore the difference of sines is just $25$.  When we set the derivative equal to zero, this factor disappears.  So now write

\[ \tan s - t = \frac{\cos s}{\cos^2 s + \sin s \sin t} \]

Let $b = \sin \alpha, c = \cos \alpha$:
\[ \tan s - t = \frac{4-xc}{(4 - xc)^2 + (31 - xb)(6 - xb)} \]
Divide by the factor of $4 - xc$ 
\[ \tan s - t = \frac{1}{(4 - xc) + (31 - xb)(6 - xb)/(4 - xc)} \]
Can we do anything with
\[ \frac{(31 - xb)(6 - xb)}{(4 - xc)} \]
Looks like a mess.

\subsection*{folded paper}

Consider a piece of paper with the dimensions $6 \times 12$.  We pick up the lower right-hand corner and place it against the left side, folding to form a crease.

\includegraphics [scale=0.4] {folded_paper1.png}
\includegraphics [scale=0.4] {folded_paper2.png}

The possible positions on the left-hand side to place that corner range from the bottom up to a distance 6 inches above the bottom.  The length of the crease is a variable, and and we wish to find the crease with the minimum length.

The variable distances can be labeled as shown on the right.  The length $L^2 = x^2 + y^2$.

I notice that the drawing is not properly scaled (the long dimension is too short).  In fact, the angle that the folded part makes on the left-hand side is a right angle.  After all, it is a corner of the original sheet, but also we have two congruent triangles, they must both be right triangles.

\begin{center} \includegraphics [scale=0.4] {folded_paper3.png} \end{center}
At this point, I notice that the problem can be re-scaled so the length of the paper is $2$ and the width is $1$, in order to simplify the arithmetic.  I have not re-done the drawings to reflect that yet, but our math will take advantage of it.

The range of the distance $x$ is $[1/2,1]$, while that for $y = [1,2]$.

We need to find a relationship between $x$ and $y$.  Start by labeling another variable distance $w$, as shown above.  

The connection that we need can be found by relating $x$, $y$ and $w$ to the total area of the paper.  We have two right triangles with sides $x$ and $y$ and total area $xy$.  

The other triangle has
\[ w^2 = x^2 - (1 - x)^2 \]
\[ = 2x - 1 \]
\[ w = \sqrt{2x - 1} \]
and area (we leave $w$ as it is for now).
\[ \frac{1}{2} (1-x) w  \]
We will need it later, so let's get the derivative of $w$ with respect to $x$
\[ \frac{dw}{dx} = \frac{1}{\sqrt{2x - 1}} = \frac{1}{w} \]

Last, we have a rhombus.  The average of the two vertical sides is 
\[ \frac{1}{2} (2 - w + 2 - y) = 2 - \frac{1}{2} (w + y) \]
and since the horizontal side is $1$, this is also equal to the area.

\subsection*{area calculation}

From the dimensions of the paper, the total area is $2$ and this is equal to the three triangles and the rhombus added together
\[ 2 = xy + \frac{1}{2} (1-x) w  + 2 - \frac{1}{2} (w + y)  \]
\[ 4 = 2xy + (1-x)w  + 4 - w - y \]
Cancel the $4$ and gather terms with $y$.  The left-hand side is
\[ y - 2xy =  -y(2x - 1) = -yw^2 \]
So
\[ -yw^2 = (1-x)w - w \]
Factor out one $w$
\[ -yw = (1 - x) - 1 = -x \]
\[ y = \frac{x}{w} \]

That's a nice simplification.  We can check that this is correct at one extreme.  When $x = w = 1$ the ratio is $1 = y$ and we see that is correct for the fold at 45 degrees.  At the other extreme we have $w = 0$ and the ratio is undefined.

And only now do I see that this calculation was unnecessary.  Draw the horizontal

\begin{center} \includegraphics [scale=0.4] {folded_paper4.png} \end{center}

Can you see the similar triangles?  The angle between $w$ and $x$ is rotated by 90 degrees counter-clockwise to form the angle between the horizontal of length $1$ and $y$, so $w/x = 1/y$, which is exactly what we said.

Now, minimize $L$
\[ L = x^2 + y^2 \]
\[ = x^2 + \frac{x^2}{w^2} \]
Take the derivative:
\[ \frac{dL}{dx} = 2x + \frac{2xw^2 - 2wx^2 \cdot 1/w}{w^4} \]

\[ \frac{dL}{dx} = 2x + \frac{2x (2x - 1) - 2x^2}{(2x -1)^2} \]
Factor out $2x$ and set equal to zero:
\[0 = 1 + \frac{(2x - 1) - x}{(2x -1)^2} \]
\[ 0 = (2x - 1)^2 + x - 1 \]
\[ 4x^2 - 3x = 0 \]
Factor out another $x$
\[ 4x - 3 = 0 \]
\[ x = \frac{3}{4} \]
The minimum crease length occurs when $x$ is halfway along its range.

I found the last problem here:

\url{https://www.math.ucdavis.edu/~kouba/CalcOneDIRECTORY/maxmindirectory/MaxMin.html}


\end{document}  