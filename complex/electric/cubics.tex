\documentclass[11pt, oneside]{article}   	% use "amsart" instead of "article" for AMSLaTeX format
\usepackage{geometry}                		% See geometry.pdf to learn the layout options. There are lots.
\geometry{letterpaper}                   		% ... or a4paper or a5paper or ... 
%\geometry{landscape}                		% Activate for for rotated page geometry
%\usepackage[parfill]{parskip}    		% Activate to begin paragraphs with an empty line rather than an indent
\usepackage{graphicx}				% Use pdf, png, jpg, or eps with pdflatex; use eps in DVI mode
								% TeX will automatically convert eps --> pdf in pdflatex	
\usepackage{amssymb}
\usepackage{amsmath}
\usepackage{parskip}
\usepackage{color}

\title{Cubics and complex numbers}
%\author{The Author}
\date{}							% Activate to display a given date or no date

\begin{document}
\maketitle
%\section{}
%\subsection{}
\Large
\noindent
Among the applications of complex numbers that are often cited is their use in solving cubic equations.  I find the procedures for solving cubics difficult to grasp, but I discovered another approach that makes sense to me, building up to cubics from quadratics.  So let's start with
\[ f(x) = ax^2 + bx + c \]
As you know, the graph of this function is a parabola, pointing up or down depending on the sign of $a$.  It's symmetric about the vertex, which can be found in various ways, but using calculus, it is the place where the slope is equal to zero
\[ 2ax + b = 0 \]
\[ x = - \frac{b}{2a} \]
\[ y = a(- \frac{b}{2a})^2 + b(- \frac{b}{2a}) + c \]
\[ y = \frac{b^2}{4a} - \frac{b^2}{2a} + c = - \frac{b^2}{4a} + c\]
Now, if the sign of $a$ is positive and $y < 0$ at the vertex, then the parabola opens up and the vertex is below the $x$-axis and there will be two real roots.  There are two places where $y=0$ and the graph crosses the $x$-axis.  If $y=0$ then there is a single real root, that is repeated.  And if $a < 0$ and $y$ is above the $x$-axis, there are no real roots.

We obtain the quadratic equation by completing the square.
\[ y = ax^2 + bx + c \]
When is $y=0$?
\[ 0 = ax^2 + bx + c \]
\[ -\frac{c}{a} = x^2 + \frac{b}{a}x  \]
\[ -\frac{c}{a} + \frac{b^2}{4a^2} = (x^2 + \frac{b}{a}x + \frac{b^2}{4a^2}) \]
\[ -\frac{c}{a} + \frac{b^2}{4a^2} = (x + \frac{b}{2a})^2 \]
\[ -4ac + b^2 = 4a^2(x + \frac{b}{2a})^2 \]
\[ \sqrt{b^2 - 4ac} = 2ax + b \]
\[ x = \frac{-b \pm \sqrt{b^2 - 4ac}}{2a} \]

As an aside, if $a=1$, then this is the same as
\[ x = \frac{b}{2} \pm \sqrt{(\frac{b}{2})^2 - c} \]
and of course, we can always divide the above equation by $a$ to achieve this, changing the coefficients $b,c$ to $b',c'$.

The term $b^2-4ac$ in the first version is called the discriminant, $D$.  If $D<0$ there are no real solutions, but we can get two complex solutions which have the form
\[ x = p \pm i q \]
To see why this is so, suppose $b^2 < 4ac$.  Factor out $-1$ to obtain $(-1)(b^2 - 4ac)$.  Then the factor of $-1$ can come out as $i$ and for the square root part we have $\pm i \sqrt{4ac - b^2}$.

These two solutions $x = p \pm i q$ are complex conjugates, so that
\[ (p + iq)(p - iq) = p^2 + q^2 \]

On the other hand, if $b^2 = 4ac$, there is only a single solution yielding $y=0$ and notice that from above, the $y$-value of the vertex of the parabola is 
\[ y = - \frac{b^2}{4a} + c \]
\[ 0 = - \frac{b^2}{4a} + c \]
\[ b^2 = 4ac \]
and then $D$ must be equal to zero.

Finally, if $D>0$, there are two real solutions.
\subsection*{cubics}
What about cubics?  Well, any cubic has an $x^3$ in it.  That means that as $x$ gets large and positive $f(x)$ is also large and positive, while if $x$ is large and negative, $f(x)$ is large and negative.  Consequently, the graph must cross the $x$-axis at least once, and so there must be at least one real root.  

When we are building a cubic from a quadratic, no matter what the quadratic, the last step must be to multiply by $(x-r)$, where $r$ is real.

So in other words, these are the roots of any cubic
\[ (x-r)(x + \sqrt{D})(x - \sqrt{D}) \]
if $D<0$ then the latter two factors are complex conjugates.  They must be, so that their product gives a completely real quadratic.  If $D=0$ then we just have
\[ (x-r)x^2 \]
This has three roots, at $x=0$ (repeated) and at $x=r$.  It turns out that the graph does not cross the $x$-axis at $x=0$, because the first derivative
\[ 2x \]
is equal to zero at $x=0$, it is a local maximum or local minimum.






\end{document}  