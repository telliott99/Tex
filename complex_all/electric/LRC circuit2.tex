\documentclass[11pt, oneside]{article}   	% use "amsart" instead of "article" for AMSLaTeX format
\usepackage{geometry}                		% See geometry.pdf to learn the layout options. There are lots.
\geometry{letterpaper}                   		% ... or a4paper or a5paper or ... 
%\geometry{landscape}                		% Activate for for rotated page geometry
%\usepackage[parfill]{parskip}    		% Activate to begin paragraphs with an empty line rather than an indent
\usepackage{graphicx}				% Use pdf, png, jpg, or eps with pdflatex; use eps in DVI mode
								% TeX will automatically convert eps --> pdf in pdflatex	
\usepackage{amssymb}
\usepackage{amsmath}
\usepackage{parskip}
\usepackage{color}

\title{LRC Circuit 2}
%\author{The Author}
\date{}							% Activate to display a given date or no date

\begin{document}
\maketitle
%\section{}
%\subsection{}
\Large
\noindent
The idea of this short write-up is to look at the equations developed in Halliday and Resnick for LRC circuit and compare their solution to what we obtained in the case of a mass and spring system with friction.

There are three types of components:  resistor with an impedance of
\[ V = iR \]
capacitor with capacitance $C$
\[ \frac{1}{C} q = V \]
\[ \frac{1}{C} \dot{q} = \frac{1}{C} i = \dot{V} \]
inductor with $L$
\[ V = L \dot{i} \]
I'm not yet sure why these make sense.  Nevertheless, they write
\[ U = \frac{1}{2}Li^2 + \frac{1}{2C}q^2\]
\[ \dot{U} = L i \ \dot{i} + \frac{1}{C} q \ \dot{q} = -i^2 R \]
Since $i = \dot{q}$, we can lose one factor of $i$
\[ L \ \dot{i} + \frac{1}{C} q = -i R \]
And since $i = \dot{q}$, $\dot{i} = \ddot{q}$:
\[ L \ \ddot{q} + \frac{1}{C} q = -\dot{q} R \]
Rearrange:
\[ L \ \ddot{q} + \dot{q} R + \frac{1}{C} q = 0 \]
\[ \ddot{q} + \frac{R}{L}\dot{q} + \frac{1}{LC} q = 0 \]

Compare to what we had for the mass and spring with a small amount of friction:
\[ \ddot{x} + \gamma \dot{x} + \omega_0{}^2 x = 0 \]
\[ \omega_0{}^2 = k/m \sim \frac{1}{LC} \]
\[ \omega' = \sqrt{\omega_0{}^2 - (\frac{\gamma}{2})^2} \sim \sqrt{\frac{1}{LC} - (\frac{R}{2L})^2}\]
\[ x(t) = e^{(-\gamma/2) t} \ C \cos(\omega't + \phi) \]
Write
\[ q(t) = q_m e^{-(R/2L) t} \ \cos \omega' t \]
Looks like the same solution to me.

\end{document}  