\documentclass[11pt, oneside]{article}   	% use "amsart" instead of "article" for AMSLaTeX format
\usepackage{geometry}                		% See geometry.pdf to learn the layout options. There are lots.
\geometry{letterpaper}                   		% ... or a4paper or a5paper or ... 
%\geometry{landscape}                		% Activate for for rotated page geometry
%\usepackage[parfill]{parskip}    		% Activate to begin paragraphs with an empty line rather than an indent
\usepackage{graphicx}				% Use pdf, png, jpg, or eps� with pdflatex; use eps in DVI mode
								% TeX will automatically convert eps --> pdf in pdflatex		
\usepackage{amssymb}
\usepackage{amsmath}
\usepackage{parskip}
\usepackage{url}

\title{Public Key Cryptography: key formats}
%\author{The Author}
\date{}							% Activate to display a given date or no date
\graphicspath{{/Users/telliott_admin/Dropbox/Tex/png/}}

\begin{document}
\maketitle
%\section{}
%\subsection{}
\Large
In this short write-up we will take a quick look at the insides of a public key cryptography key pair.  I will use the Python \texttt{rsa} module to generate very short keys to make it easier to see what's what.  These are what we previously called type C \texttt{pkcs1} keys.

Start with the following values

\texttt{e = 17 \\
p = 151 \\
q = 211 \\
n = p*q  \# 31861 \\
phi = (p-1)*(q-1)  \# 31500 \\
d = modinv.modinv(e,phi)  \# 1853 \\
d*e \% phi  \# 1 }

I found the value of $d$ as described in a previous write-up of this series.

In Python:

\texttt{>>> import rsa \\
>>> pbk = rsa.PublicKey(n=31861,e=17) \\
>>> pbk \\
PublicKey(31861, 17) \\
>>> data = pbk.save\_pkcs1() \\
>>> b64 = data.split(NL)[1] \\
>>>}

Here I have defined \texttt{NL} to be the standard Unix newline, which I'm having trouble representing in this document.  Now we decode the base64

\texttt{>>> b64 \\
u'MAcCAnx1AgER' \\
>>> import base64 \\
>>> h = base64.b64decode(b64) }

I can't show you the original hex either.  But I can do this:

\begin{minipage}{4in}\tt
\begin{tabbing}
xxxx\=xxxx\=xxxx\=xxxx\=xxxx\=\kill
>>> import struct \\
>>> for c in h: \\
...\> print hex(ord(c)), \\
...  \\
0x30 0x7 0x2 0x2 0x7c 0x75 0x2 0x1 0x11 \\
\end{tabbing}
\end{minipage}

An alternative, equivalent way would be:

\texttt{
>>> t = (struct.unpack("B",c)[0] for c in h) }

The advantage of \texttt{struct.unpack} is it allows you to do multibyte representations like \texttt{int64}.

The value of $t$ is \texttt{48, 7, 2, 2, 124, 117, 2, 1, 17}.

The way the layout works is that each region consists of a byte describing the type of value, then the size of the value, then the value itself.  For example, \texttt{2,2,124,117} is a value of type integer of length 2 bytes, and the value is \texttt{124*256 + 117} which is equal to

\texttt{>>> 124*256 + 117 \\
31861}

This is just $n$.  Similarly, \texttt{2,1,17} is $e$.  The leading terms \texttt{48,7}  obviously describe the format.

If we generate a private key

\texttt{pk = rsa.PrivateKey(n=31861,e=17,p=151,q=211,d=1853)}

and export the data in the same way, the base64 is

\texttt{>>> b64
u'MCACAQACAnx1AgERAgIHPQICAJcCAgDTAgE1AgIArQIBSQ=='}

And the decoded ints are

\texttt{48 32} header

\texttt{2 1 0}  value \text{0} ??

\texttt{2 2 124 117} $n$, as before

\texttt{2 1 17} $e$, as before

\texttt{2 2 7 61} This is \texttt{1853}, that is, $d$.

\texttt{2 2 0 151} $p$

\texttt{2 2 0 211} $q$

\texttt{2 1 53}

\texttt{2 2 0 173}

\text{2 1 73}

The last three values are 53, 173 and 73.  I believe from looking at the XML format that these are DP, DQ and InverseQ, but I am not sure yet what they are or how they are used.

\subsection{}

As long as we're at it, let's look at what we called type A \texttt{ssh-rsa} keys.  This is the type that I analyzed in my blog posts.

\url{telliott99.blogspot.com/2011/08/dissecting-rsa-keys-in-python-2.html}

I split off the first and the last part (separated by spaces) by hand.

In Python:

\texttt{>>> import utils \\
>>> data = utils.load\_data("kf.pub") \\
>>> data \\
'ssh-rsa AAAAB3NzaC1yc2EAAAADAQABAAABAQC8u8w9.. \\
>>> data = data.split(" ")[1] \\
>>> data \\
'AAAAB3NzaC1yc2EAAAADAQABAAABAQC8u8w9K4aRPglzdPj.. \\
>>> import base64 \\
>>> s = base64.b64decode(data) \\
>>> fn = "out.txt" \\
>>> FH = open(fn,"wb") \\
>>> FH.write(s) \\
>>> FH.close()}

\large
\texttt{\$ hexdump -C out.txt \\
00000000  00 00 00 07 73 73 68 2d  72 73 61 00 00 00 03 01  |....ssh-rsa.....|
.. }
\Large

ssh-rsa again!

\texttt{00 00 00 03 is a size, 3 bytes \\
01 00 01 is an integer = 1 * 16**4 + 1 = 65537 \\
00 00 01 01 is a size, 257 bytes}

that's how many there are starting from 00 bc

In Python, again:

\texttt{>>> data = load\_data('out.txt') \\
>>> data = data[22:] \\
>>> len(data) \\
257 \\
>>> L = list(data) \\
>>> L.reverse() \\
>>> import struct \\
>>> L2 = [struct.unpack("B",c)[0] for c in L] \\
>>>}

\texttt{>>> b = 256 \\
>>> n = 0 \\
>>> for i,x in enumerate(L2): \\
...     x *= b**i \\
...     n += x \\
...  \\
>>> n \\
23825407884424843043892774272494727..}

check it with rsa

\texttt{>>> import rsa \\
>>> k = utils.load\_data('kf') \\
>>> pk = rsa.PrivateKey.load\_pkcs1(k) \\
>>> pk.n \\
23825407884424843043892774272494727.. \\
>>> n == pk.n
True \\
>>> }

It matches!


\end{document}  