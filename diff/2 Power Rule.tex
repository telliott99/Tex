\documentclass[11pt, oneside]{article}   	% use "amsart" instead of "article" for AMSLaTeX format
\usepackage{geometry}                		% See geometry.pdf to learn the layout options. There are lots.
\geometry{letterpaper}                   		% ... or a4paper or a5paper or ... 
%\geometry{landscape}                		% Activate for for rotated page geometry
%\usepackage[parfill]{parskip}    		% Activate to begin paragraphs with an empty line rather than an indent
\usepackage{graphicx}				% Use pdf, png, jpg, or eps� with pdflatex; use eps in DVI mode
								% TeX will automatically convert eps --> pdf in pdflatex		
\usepackage{amssymb}

\title{Derivatives}
%\author{The Author}
\date{}							% Activate to display a given date or no date

\begin{document}
\maketitle
%\section{}
%\subsection{}
\large
The basic method for finding the slope of a tangent line to a function $f(x)$ at $x$ is to compute the difference quotient
\[ \frac{f(x+h) - f(x)}{h} \]
for a small change $h$.  Since $ h = \Delta x$ and  $ \Delta y = f(x + h) - f(x)$, this is $\Delta y / \Delta x$, the slope of a secant line between the points $(x,f(x))$ and $(x+h, f(x+h))$ on the curve $y=f(x)$.  To find the derivative we find the limit
\[ lim_{h \to 0}  \  \frac{f(x+h) - f(x)}{h}  \]
Last time we went through some examples

\[ f(x) = x^2 \ \ \Rightarrow \ \  f'(x) = 2x \]
\[ f(x) = \sqrt{x} = x^{1/2}\ \ \Rightarrow \ \  f'(x) = \frac{1}{2}x^{-1/2}\]
\[ f(x) = \frac{1}{x} = x^{-1} \ \ \Rightarrow \ \  f'(x) = -\frac{1}{x^2} = -x^{-2} \]
We deduce that the general formula is
\[ f(x) = x^n \ \ \Rightarrow \ \  f'(x) = nx^{n-1} \]

We would like to find a general expression for $f'(x)$, when $f(x) = x^n$ with integer $n$ ($n = 1, 2, 3 \cdots$).  Recall that the Binomial Theorem gives the expansion of $(x + h)^n$.  We only need the first three terms.
\[ (x + h)^n = x^n + nx^{n-1}h + n(n-1)x^{n-2}h^2 + \cdots  \]
Now we compute the difference quotient and find the limit
\[ \frac{f(x+h) - f(x)}{h} \]
\[ \frac{x^n + nx^{n-1}h + n(n-1)x^{n-2}h^2 + \cdots - x^n}{h} \]
\[ \frac{nx^{n-1}h + n(n-1)x^{n-2}h^2 + \cdots} {h} \]
\[ nx^{n-1} + n(n-1)x^{n-2}h + \cdots  \]
 and find the limit
 \[ lim_{h \to 0}  \ nx^{n-1} + n(n-1)x^{n-2}h = nx^{n-1}  \]
 This is called the power rule
 \[f(x) = x^n, \ \ \ \ f'(x) = nx^{n-1} \]
 
Another question is what to do with a sum or difference of polynomials, such as 
 \[ f(x) + g(x) \]
 If you write out the difference quotient in the second case
 \[ \frac{ f(x+h) - f(x) + g(x+h) - g(x)}{h} \]
 everything can be exactly as before, just grouping all the terms from $f(x)$ and those from $g(x)$ separately. 
 
\[  [f(x) + g(x)]' = f'(x) + g'(x)  \]

\end{document}  