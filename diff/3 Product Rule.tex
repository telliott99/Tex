\documentclass[11pt, oneside]{article}   	% use "amsart" instead of "article" for AMSLaTeX format
\usepackage{geometry}                		% See geometry.pdf to learn the layout options. There are lots.
\geometry{letterpaper}                   		% ... or a4paper or a5paper or ... 
%\geometry{landscape}                		% Activate for for rotated page geometry
%\usepackage[parfill]{parskip}    		% Activate to begin paragraphs with an empty line rather than an indent
\usepackage{graphicx}				% Use pdf, png, jpg, or eps� with pdflatex; use eps in DVI mode
								% TeX will automatically convert eps --> pdf in pdflatex		
\usepackage{amssymb}

\title{Product, Chain and Quotient Rules}
%\author{The Author}
\date{}							% Activate to display a given date or no date

\begin{document}
\maketitle
%\section{}
%\subsection{}
\large
In previous work, we found out how to differentiate (find the derivative of) polynomial functions of x.  The power rule is:
\[ f(x) = x^n, \ \ \ \  f'(x) = nx^{n-1} \]
We also saw that constants just come along with the result 
\[ f(x) = ax^n, \ \ \ \  f'(x) = anx^{n-1} \]
We didn't bother to show it, but you already know that
\[ f(x) = a, \ \ \ \  f'(x) = 0  \]
The slope of $y = constant$ is just 0.
And we saw that the power rule extends to negative and rational exponents.
\[ f(x) = \frac{1}{x}, \ \ \ \  f'(x) = -\frac{1}{x^2} \]
\[ f(x) = \sqrt{x}, \ \ \ \  f'(x) = \frac{1}{2\sqrt{x}}  \]
Finally, we saw that for a sum of two functions
\[ f(x) + g(x), \ \ \ \  (f(x) + g(x))' = f'(x) + g'(x)  \]
This leads naturally to the  question, what about a product of functions?
\[ f(x) \ g(x), \ \ \ \  (f(x) \ g(x))' = ?  \]
I won't go through a proof of this, but the product rule is
\[ (f(x) \ g(x))' = f(x) \ g'(x) + f'(x) \ g(x)  \]
Written with differentials and different letters $u$ and $v$ this is the same as
\[ u = f(x), v = g(x) \]
\[ \frac{d}{dx}(uv) = u \frac{dv}{dx} + v \frac{du}{dx} \]
which is frequently abbreviated as
\[ (uv)' = u v' + v u'  \]
I find this form particularly easy to remember, but don't forget what $v'$ means.  It means $\frac{dv}{dx}$ (or some other variable).
Sometimes the product rule is given as "this times the derivative of that plus that times the derivative of this", where "this" is $u$ and "that" is $v$.

Let's do some examples
\[ f(x) = x^2 = x \ x, \ \ \ \  (x \ x)' = x (1) + (1) x = 2x  \]
The answer must match what we already know to be true!
\[ f(x) = \sqrt{x} \sqrt{x}, \ \ \ \ (\sqrt{x} \sqrt{x})' = \frac{1}{2\sqrt{x}} \sqrt{x} + \sqrt{x} \frac{1}{2\sqrt{x}} = 1\]
\[ f(x) = ax, \ \ \ \ (ax)' = a \ (1) + (0) x = a \]
A last one that we already know.  Suppose 
\[ f(x) = (x+3)(2x-2) = 2x^2 + 4x - 6 \]
by the power rule:
\[ f'(x) = \frac{d}{dx} 2x^2 + 4x - 6 = 4x + 4\]
by the product rule:
\[ f'(x) = \frac{d}{dx} (x+3)(2x-2) = (x+3)(2) + (2x-2)(1) = 4x + 4 \]


It is worth while to play around with some examples, once we know the trig function derivatives, and add the exponentials when we get there.
\[ f(x) = x \ sin \ x, \ \ \ \  f'(x) = x \ cos \ x + sin \ x \]
\[ f(x) = sin x \ cos \ x, \ \ \ \  f'(x) = -sin^2x + cos^2x = 2 cos^2x - 1 \]
Later, when we want to go backward from the derivative $cos^2x$ to the original function (i.e. integration), this second result will become very useful.

\end{document}  