\documentclass[11pt, oneside]{article}   	% use "amsart" instead of "article" for AMSLaTeX format
\usepackage{geometry}                		% See geometry.pdf to learn the layout options. There are lots.
\geometry{letterpaper}                   		% ... or a4paper or a5paper or ... 
%\geometry{landscape}                		% Activate for for rotated page geometry
%\usepackage[parfill]{parskip}    		% Activate to begin paragraphs with an empty line rather than an indent
\usepackage{graphicx}				% Use pdf, png, jpg, or eps� with pdflatex; use eps in DVI mode
								% TeX will automatically convert eps --> pdf in pdflatex		
\usepackage{amssymb}
\usepackage{amsmath}
\usepackage{parskip}

\graphicspath{{/Users/telliott_admin/Dropbox/Tex/png/}}

\title{Euler's Gem}
%\author{The Author}
\date{}							% Activate to display a given date or no date

\begin{document}
\maketitle
%\section{}
%\subsection{}
\Large

Here are sketches of two different derivations of Euler's famous formula, both following Dunham's book about Euler$^*$

\[ e^{i\theta} = \cos \theta + i \sin \theta \]

And of course, if $\theta = \pi$, we have

\[ e^{i\pi} = -1 + 0 \]
\[ e^{i\pi}+ 1 = 0 \]

(what Feynman called "our jewel").

Using $x$ is a bit simpler notation, so that's what I'll do here
\[ e^{ix} = \cos x + i \sin x \]

\subsection*{preliminary}

Start with the definition of $i$
\[ i = \sqrt{-1} \]

Having $i$ gives us new factorizations like
\[ a^2 + b^2 = (a + bi)(a - bi) \]
since the terms with $\pm \ abi$ cancel and $- i^2 = 1$.  
So
\[ 1 = \cos^2 x + \sin^2 x \]
\[ 1 = (\cos x + i \sin x) (\cos x - i \sin x) \]
(Of course, we could switch sine and cosine here, but this is the convention.

Below, we will need the above plus one more identity involving $i$:
\[ -i^2 = 1 \]
so
\[ u = - i^2 u \]
\[ \frac{u}{i} = - i u \]

\subsection*{number one}

Start with the inverse sine function:
\[ x = \sin^{-1} y \]
\[ y = \sin x \]
\[ dy = \cos x \ dx \]

Then the side adjacent to $x$ is $\sqrt{1-y^2}$ and so

\[ \cos x = \sqrt{1-y^2} \]

We're interested in the integral
\[ \int \frac{1}{\sqrt{1-y^2}} \ dy \]
which is just
\[ = \int \frac{1}{\cos x} \ \cos x \ dx = x \]

Now, Euler makes a complex change of variable
\[ y = iz \]
\[ \frac{1}{1-y^2} = \frac{1}{1+z^2} \]
\[ x = \int \frac{1}{1 - y^2} \ dy \]
\[ = \int \frac{1}{\sqrt{1 + z^2}} \ i \ dz \]
we have converted the integral to having a plus sign under the square root and the answer is

\[ = i \ln \ (\sqrt{{1 + z^2}} + z) \]

(I will justify this elsewhere---it's a standard trig substitution but a bit complicated).

Now, just undo the substitution:
\[ z = \frac{y}{i} = \frac{ \sin x}{i} \]
\[ \sqrt{1 + z^2} = \sqrt{1 - y^2} = \cos x \]
Hence our previous result
\[ x = i \ln \ (\sqrt{{1 + z^2}} + z) \]
is equivalent to
\[ x = i \ln \ (\cos x + \frac{\sin x}{i}) \]

Recall our two identities involving $i$.  The first one was
\[ \frac{u}{i} = - i u \]
So when we had:
\[ x = i \ln \ (\cos x + \frac{\sin x}{i}) \]
\[ x = i \ln \ (\cos x - i \sin x) \]
\[ ix = - \ln \ (\cos x - i \sin x) \]
\[ = \ln \ \frac{1}{(\cos x - i \sin x)} \]
again
\[ \frac{1}{\cos u - i \sin u} = \cos u + i \sin u \]
so
\[ ix = \ln \ \frac{1}{(\cos x - i \sin x)}  = \ln \ (\cos x + i \sin x) \]
Just exponentiate:
\[ e^{ix} = \cos x + i \sin x \]

\subsection*{number two}

Suppose we try this multiplication:
\[ (\cos s + i \sin s) (\cos t + i \sin t) \]
\[ = \cos s \cos t + i \sin s \cos t + i \cos s \sin t - \sin s \sin t \]
\[ = (\cos s \cos t - \sin s \sin t) + i (\sin s \cos t + \cos s \sin t) \]
\[ = \cos (s+t) + i \sin(s + t) \]

set $s=t$
\[ (\cos s + i \sin s)^2 = \cos 2s + i \sin 2s \]
In fact, Euler showed that it works for fractional $n$ but I'll assume that part.
\[ (\cos s + i \sin s)^n = \cos ns  + i \sin ns \]

Now multiply the difference rather than the sum:
\[ (\cos s - i \sin s) (\cos t - i \sin t) \]
\[ = (\cos s \cos t - \sin s \sin t ) - i (\sin s \cos t + \sin t \cos s) \]
\[ = \cos(s+t) - i (\sin (s + t)) \]

again, with $s=t$
\[ ( \cos s - i \sin s)^2 = \cos 2s - i sin 2s \]
\[ ( \cos s - i \sin s)^n = \cos ns - i sin ns \]

Restate the two results:
\[ (\cos s + i \sin s)^n = \cos ns  + i \sin ns \]
\[ ( \cos s - i \sin s)^n = \cos ns - i sin ns \]
Add them
\[ 2 \cos ns = (\cos s + i \sin s)^n + ( \cos s - i \sin s)^n \]

\subsection*{where the magic happens}

Let 
\[ s = \frac{x}{n} \]
As $n \rightarrow \infty$, $s \rightarrow 0$, and 
\[ \sin s \rightarrow s \] 
\[ \cos s \rightarrow 1\]

\[ \cos x  = \cos ns \]
\[ = \frac{1}{2} \ [ \  (\cos s + i \sin s)^n + ( \cos s - i \sin s)^n \ ] \]
\[ = \frac{1}{2} \  [ \ (1 + i s)^n + (1 - i s)^n \ ] \]
\[ = \frac{1}{2} \ [ \ (1 + \frac{ix}{n})^n + (1 - \frac{ix}{n})^n \ ]  \]
but
\[ e^{ix} = (1 + \frac{ix}{n})^n \]
hence
\[  \cos x  = \frac{1}{2} \ [ \ e^{ix} + e^{-ix} \ ] \]

By very similar manipulation to what's in the first part we can also obtain an expression for the sine:

\[ 2i \ \sin(ns) = (\cos s + i \sin s)^n - (\cos s - i \sin s)^n \]
which will lead to
\[ \sin x = \frac{1}{2i} \ (e^{ix} - e^{-ix}) \]

Adding together
\[ 2(\cos x + i \sin x) = e^{ix} + e^{-ix} + e^{ix} - e^{-ix} \]
\[ \cos x + i \sin x = e^{ix} \]

\subsection*{check}

Before quitting, we should check the formula.  One way is to notice the connection between infinite series expansions for $e^x$:

\[ e^x = \sum_{n=0}^{\infty} \frac{x^n}{n!}  =  1 + x + \frac{x^2}{2!} + \frac{x^3}{3!} + \frac{x^4}{4!} \dots \]

and sine:
\[ \sin x = x - \frac{x^3}{3!} + \frac{x^5}{5!} - \frac{x^7}{7!} \dots \]

and cosine:
\[ \cos x = 1 - \frac{x^2}{2!} + \frac{x^4}{4!} - \frac{x^6}{6!} \dots \]

These can almost be added together to give what we seek, except for the problem of the alternating signs.  What happens with $e^{ix}$?

\[ e^{ix} =  1 + ix + \frac{i^2x^2}{2!} + \frac{i^3x^3}{3!} + \frac{i^4x^4}{4!} \dots \]
\[ = 1 + ix - \frac{x^2}{2!} - i  \frac{x^3}{3!} + \frac{x^4}{4!} \dots \]

The pattern is 
\[  \sum_{n=0}^{\infty} i^n = 1 + i - 1 - i + 1 \dots \]

And we're there.  We just have to recognize that the pattern with $e^{ix}$ has $i \sin x$ so as we said
\[ e^{ix} = \cos x + i \sin x \]




\end{document}  