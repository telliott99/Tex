\documentclass[11pt, oneside]{article}   	% use "amsart" instead of "article" for AMSLaTeX format
\usepackage{geometry}                		% See geometry.pdf to learn the layout options. There are lots.
\geometry{letterpaper}                   		% ... or a4paper or a5paper or ... 
%\geometry{landscape}                		% Activate for for rotated page geometry
%\usepackage[parfill]{parskip}    		% Activate to begin paragraphs with an empty line rather than an indent
\usepackage{graphicx}				% Use pdf, png, jpg, or eps� with pdflatex; use eps in DVI mode
								% TeX will automatically convert eps --> pdf in pdflatex		
\usepackage{amssymb}

\title{Logarithms:  Change of Base}
%\author{The Author}
\date{}							% Activate to display a given date or no date

\begin{document}
\maketitle
%\section{}
%\subsection{}
\large
The formula for change of base is
\Large \[ log_a(x) = log_b(x)/log_b(a) \]
\large
The way I use to remember this is to say to myself that $log_a(x)$ and $log_b(x)$ are related somehow---that's what we want---so they are on opposite sides of the equation.  Then, there is a ratio with another log in the denominator, and \emph{that log must be to the same base as in the numerator}.
You can check it with easy bases like $2$ and $4$
\Large
\[ log_2(16) = 4 \]
\[ log_4(16) = 2 \]
\[ log_4(2) = 1/2 \]
\large 
so
\Large
\[ log_2(16) = log_4(16)/log_4(2) \]
\large
is correct.
Here is a simple derivation:
\Large
\[ x = b^y \]
\[ y = log_b(x) \]
\[ log_a(x) = log_a(b^y) = y \ log_a(b) = log_b(x) \ log_a(b) \]
\[ log_a(x) / log_a(b) = log_a(b) \]

\end{document}  