\documentclass[11pt, oneside]{article}   	% use "amsart" instead of "article" for AMSLaTeX format
\usepackage{geometry}                		% See geometry.pdf to learn the layout options. There are lots.
\geometry{letterpaper}                   		% ... or a4paper or a5paper or ... 
%\geometry{landscape}                		% Activate for for rotated page geometry
%\usepackage[parfill]{parskip}    		% Activate to begin paragraphs with an empty line rather than an indent
\usepackage{graphicx}				% Use pdf, png, jpg, or eps� with pdflatex; use eps in DVI mode
								% TeX will automatically convert eps --> pdf in pdflatex		
\usepackage{amssymb}
\usepackage{amsmath}
\usepackage{parskip}

\title{problem ?}
%\author{The Author}
%\section{}
% \subsection*{R code}
\date{}							% Activate to display a given date or no date

\graphicspath{{/Users/telliott_admin/Dropbox/Tex/png/}}

\begin{document}
\maketitle
\large
%\noindent
\[ \int \frac{\sin 2x}{1 + \cos^2 x} \ dx \]
The formulas to remember are:
\[ \cos s + t = \cos s \cos t - \sin s \sin t \]
\[ \sin s + t = \sin s \cos t + \cos s \sin t \]
If $s=t$ the second one becomes:
\[ \sin 2s = 2 \sin s \cos s \]
So our problem is now:
\[ \int \frac{2 \sin x \cos x}{1 + \cos^2 x} \ dx \]
Notice that
\[ \frac{d}{dx} \cos^2 x = - 2 \cos x \sin x \ dx \]
That leads to the idea of letting
\[ u = 1 + \cos^2 x \]
\[ du = - 2 \cos x \sin x \ dx \]
So the integral is 
\[ = \int \frac{-du}{u} \]
\[ = - \ln u \]
\[ = - \ln (1 + \cos^2 x) + C \]
We don't need absolute value signs because $1 + \cos^2 \ge 0$.
Check by differentiating.  Remember the minus sign, the rest is
\[ \frac{1}{1 + \cos^2 x} \ (-2 \cos x \sin x) \]
so, with the minus sign we have our integrand back again.  Check.
\end{document}  