\documentclass[11pt, oneside]{article}   	% use "amsart" instead of "article" for AMSLaTeX format
\usepackage{geometry}                		% See geometry.pdf to learn the layout options. There are lots.
\geometry{letterpaper}                   		% ... or a4paper or a5paper or ... 
%\geometry{landscape}                		% Activate for for rotated page geometry
%\usepackage[parfill]{parskip}    		% Activate to begin paragraphs with an empty line rather than an indent
\usepackage{graphicx}				% Use pdf, png, jpg, or eps� with pdflatex; use eps in DVI mode
								% TeX will automatically convert eps --> pdf in pdflatex		
\usepackage{amssymb}
\usepackage{amsmath}
\usepackage{parskip}

\title{Integration of $\sin x \cos x$}
%\author{The Author}
%\section{}
% \subsection*{R code}
\date{}							% Activate to display a given date or no date

\graphicspath{{/Users/telliott_admin/Dropbox/Tex/png/}}

\begin{document}
\maketitle
\Large
%\noindent

Integrals involving the double-angle formula can be tricky because there may be two different answers which \emph{are both right} but can be seen to be equivalent only after some manipulation.  Consider

\[ \int \sin x \cos x \ dx \]

A simple answer is to substitute $u = \sin x$, then obviously we have $\int u \ du$ so the answer is

\[ = \frac{1}{2} \sin^2 x + C \]

which is easily checked by differentiation.  However, on an exam they may try some trickery like this with the double-angle formula:

\[ \sin x \cos x = \frac{1}{2} \sin (2x) \]

So the integral is
\[ = \int  \frac{1}{2} \sin (2x) \ dx \]
\[ = - \frac{1}{4} \cos (2x) + C \]

Is it really true that

\[ \frac{1}{2} \sin^2 x \overset{?}{=} - \frac{1}{4} \cos (2x) \]

We can show that these are equal.  The double-angle formula for cosine is:

\[ \cos (2x) = \cos^2 x - \sin^2 x \]
\[ = 1 - 2\sin^2 x \]

substituting 
\[ - \frac{1}{4} \cos (2x) + C = - \frac{1}{4}(1 - 2\sin^2 x) + C \]
\[ = -\frac{1}{4} + \frac{1}{2} \sin^2 x + C \]

And now we see that they are the same, we just have to remember that the $C$ in the first answer \emph{is not the same} as the $C$ in the second answer.

Try checking the second answer by differentiation:

\[ \frac{d}{dx} \ - \frac{1}{4} \cos (2x) \]
\[ = \frac{1}{2} \ \sin (2x) \]
\[ = \sin x \cos x \]

\subsection*{cosine squared}

We've solved this before, I thought I'd just repeat it here:

\[ \int \cos^2 x \ dx \]

Start with

\[ \cos (2x) = \cos^2 x - \sin^2 x \]
\[ = 2 \cos^2 x - 1 \]
Thus
\[ \cos^2 x = \frac{1}{2} (\cos (2x) + 1) \]

so the integral is
\[ = \frac{1}{2} \int (\cos (2x) + 1) \]
\[ = \frac{1}{2}\ (\ \frac{1}{2} \ \sin (2x) + x) + C \]

But this also has a second version.  The simplest way is just to see what happens when we differentiate

\[ \frac{d}{dx} \sin x \cos x = -\sin^2 x + \cos^2 x \]
\[ = 2 \cos^2 x - 1 \]

Hence
\[ \sin x \cos x = -x + 2 \int cos^2 x \ dx \]
\[ \int cos^2 x \ dx = \frac{1}{2} (x + \sin x \cos x) + C \]
(writing the constant now). 

Our two answers must be the same, somehow, within a constant:

\[ \frac{1}{2}\ (\ \frac{1}{2} \ \sin (2x) + x) \overset{?}{=} \frac{1}{2} (x + \sin x \cos x) \]
\[ \ \frac{1}{2} \ \sin (2x) + x \overset{?}{=} (x + \sin x \cos x) \]
\[ \frac{1}{2} \ \sin (2x) = \sin x \cos x  \]

The double-angle formula, again.




\end{document}  