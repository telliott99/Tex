\documentclass[11pt, oneside]{article}   	% use "amsart" instead of "article" for AMSLaTeX format
\usepackage{geometry}                		% See geometry.pdf to learn the layout options. There are lots.
\geometry{letterpaper}                   		% ... or a4paper or a5paper or ... 
%\geometry{landscape}                		% Activate for for rotated page geometry
%\usepackage[parfill]{parskip}    		% Activate to begin paragraphs with an empty line rather than an indent
\usepackage{graphicx}				% Use pdf, png, jpg, or eps� with pdflatex; use eps in DVI mode
								% TeX will automatically convert eps --> pdf in pdflatex		
\usepackage{amssymb}
\usepackage{amsmath}
\usepackage{parskip}
\usepackage{hyperref}
\usepackage[english]{babel}

\title{List of trig integrals}
%\author{The Author}
%\section{}
% \subsection*{R code}
\date{}							% Activate to display a given date or no date

\graphicspath{{/Users/telliott_admin/Dropbox/Tex/png/}}

\begin{document}
\maketitle
\large
\noindent

For each pair, we list $F(x)$ first and then $\int f(x) \ dx$---since it feels more natural to reconstruct each one by taking the derivative of $F(x)$.  Don't forget the constant:
\[ F(x) + C = \int f(x) \ dx \]
\[ \sin x \Longleftrightarrow \cos x \]
\[ \cos x \Longleftrightarrow -\sin x  \]
\[ \tan x \Longleftrightarrow \sec^2 x \]
\[ \sec x \Longleftrightarrow \sec x \tan x \]
The derivatives of cosecant and cotangent can be reconstructed by noticing the pattern of substituting the corresponding "co" functions and changing sign:
\[ \csc x \Longleftrightarrow - \csc x \cot x \]
\[ \cot x \Longleftrightarrow -\csc^2 x \]
We are missing tangent and secant from the list of derivatives .  The tangent is obtained by $u$-substitution (sine is \emph{minus} the derivative of cosine).  The secant is obtained by a trick, multiplying on top and bottom by $(\sec x + \tan x)$ giving $\int du/u$:
\[ -\ln |\cos x| \Longleftrightarrow \tan x \]
\[ \ln |\sec x + \tan x| \Longleftrightarrow \sec x \]
Again, the integrals of cosecant and cotangent can be constructed by substituting "co" functions and flipping signs (it is easy to check it for cotangent).
\[ \ln |\sin x| \Longleftrightarrow \cot x \]
\[ -\ln |\csc x + \cot x| \Longleftrightarrow \csc x \]
I suggest memorizing all 10 of these, and writing them out quickly at the beginning of an exam.  It provides a resource so you don't have to remember them (especially the co versions), under pressure.
\end{document}  