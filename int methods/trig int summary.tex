\documentclass[11pt, oneside]{article}   	% use "amsart" instead of "article" for AMSLaTeX format
\usepackage{geometry}                		% See geometry.pdf to learn the layout options. There are lots.
\geometry{letterpaper}                   		% ... or a4paper or a5paper or ... 
%\geometry{landscape}                		% Activate for for rotated page geometry
%\usepackage[parfill]{parskip}    		% Activate to begin paragraphs with an empty line rather than an indent
\usepackage{graphicx}				% Use pdf, png, jpg, or eps� with pdflatex; use eps in DVI mode
								% TeX will automatically convert eps --> pdf in pdflatex		
\usepackage{amssymb}
\usepackage{amsmath}
\usepackage{parskip}

\title{Summary of trig and other integrals}
%\author{The Author}
%\section{}
% \subsection*{R code}
\date{}							% Activate to display a given date or no date

\graphicspath{{/Users/telliott_admin/Dropbox/Tex/png/}}

\begin{document}
\maketitle
\Large
%\noindent

To make things simpler, I'm going to use the convention that the function $F(x)$ has as its derivative $f(x)$. 
\[ \frac{d}{dx} \ F(x) = f(x) \]
In that case, we write, alternatively
\[ \int f(x) \ dx = F(x) + C \]
For a \emph{definite} integral, we evaluate $F$ at the two endpoints of the interval $[a,b]$
\[ \int_a^b f(x) \ dx = F(b) - F(a) \]
but for the \emph{indefinite} integral we need to remember to write the constant $C$.

\subsection*{basic trig derivatives}

You've seen differentiation of the 6 basic trig functions before, namely:

\begin{flalign*}
& \ \ \ \ \ \   F(x) = \cos x \ ; \ \ \  f(x) = -\sin x & \\
& \ \ \ \ \ \   F(x) = \sin x \ ; \ \ \  f(x) = \cos x & \\
& \ \ \ \ \ \   F(x) = \tan x \ ; \ \ \  f(x) = \sec^2 x & \\
\end{flalign*}

\begin{flalign*}
& \ \ \ \ \ \   F(x) = \sec x \ ; \ \ \  f(x) = \sec x \ \tan x & \\
& \ \ \ \ \ \   F(x) = \csc x \ ; \ \ \  f(x) = - \csc x \ \cot x & \\
& \ \ \ \ \ \   F(x) = \cot x \ ; \ \ \  f(x) = - \csc^2 x & 
\end{flalign*}

Our focus is on integration, so when faced with a problem like

\[ \int \sin x \ dx = - \cos x + C \]

you just find $f(x) = -\sin x$ in the list, read off $F(x)$, and take account of the minus sign.

I'm sure you've noticed the symmetry between the "co-" functions $\csc x$ and $\cot x$ and the ones above.  Just remember the minus sign.

And don't forget, if you see this problem you should be able to give the solution:

\[ \int \sec^2 x \ dx = \tan x + C \]\

\subsection*{going from $f(x)$ to $F(x)$}

Now, looking at the list of functions $f(x)$ above, which trig functions are missing?  We need to be able to solve problems like:

\[ \int \tan x \ dx \]
\[ = \int \frac{\sin x}{\cos x} \ dx \]

Substitute $u = \cos x$ and notice that we have $-du = \sin x \ dx$ as well.  That is,
\[ \int \frac{\sin x}{\cos x} \ dx \]
\[ = -\int \frac{1}{u} \ du \]
\[ = -\ln \ |u | + C \]
\[ = -\ln \ |\cos x | + C \]
And by the usual symmetry we have a similar result for the cotangent.  

\[ \int \cot x \ dx =  \ln \ |\sin x \ | \]

What about the secant?

\[ \int \sec x \ dx  \]

This involves a more subtle trick.

\[ = \int \sec x \ \frac{\sec x + \tan x}{\sec x + \tan x} \ dx  \]
\[ = \int \ \frac{\sec^2 x + \sec x \ \tan x}{\sec x + \tan x} \ dx  \]

Notice that if $u = \sec x + \tan x$, then $du = \sec^2 x + \sec x \ \tan x \ dx$

So, substituting, all we have is just:
\[ = \int \frac{1}{u} \ du \]
\[ = \ln \ |u | + C \]
\[ = \ln \ | \sec x + \tan x \ | + C \]

Fill out our table:

\begin{flalign*}
& \ \ \ \ \ \    F(x) = \sin x \ ; \ \ \  f(x) = -\cos x  & \\
&  \ \ \ \ \ \   F(x) = \cos x \ ; \ \ \  f(x) = \sin x & \\
& \ \ \ \ \ \    F(x) = \tan x \ ; \ \ \  f(x) = -\ln \ |\cos x \ | & \\
& \ \ \ \ \ \    F(x) = \sec x \ ; \ \ \  f(x) = \ln \ | \sec x + \tan x \ |  & \\
& \ \ \ \ \ \    F(x) = \csc x \ ; \ \ \ f(x) = - \ln \ | \csc x + \cot x \ | & \\
& \ \ \ \ \ \    F(x) = \cot x \ ; \ \ \  f(x) = \ln \ |\sin x | & \\
\end{flalign*}

\subsection*{cosine squared}

One more trig function that comes up regularly is $\cos^2 x$.  I've worked it elsewhere so here, let me just list the answer:

\[ \int \cos^2 \  x \ dx = \frac{1}{2} (x + \sin x \ \cos x) + C \]

It is easily checked by differentiating $F(x)$.  There is an alternative formula:

\[ \int \cos^2 \  x \ dx = \frac{1}{2} (x + \frac{1}{2} \ \sin \ 2x) + C \]

The two are related by the double-angle formula.

\subsection*{logarithms and exponentials}

You know (and we used it above) that
\[ \int \frac{1}{x} \ dx = \ln x + C \]

But what is
\[ f(x) = \ln x \ ; \ \ \  F(x) = ?? \]

Let's try differentiating
\[ \frac{d}{dx} \ x \ln x \]
\[ = x \frac{1}{x} + \ln x = 1 +  \ln x \]
So..
\[ \frac{d}{dx} \ (x \ln x - x) =  \ln x \]

Our table should then have
\[ f(x) = \frac{1}{x} \ ; \ \ \  F(x) = \ln x \]
\[ f(x) =  \ln x \ ; \ \ \  F(x) = \ x \ln x - x \]

We can also try differentiating another product

\[ \frac{d}{dx} \ x e^x \]
\[ = x e^x + e^x \]
So
\[ \frac{d}{dx} \ (x e^x - e^x )  = x e^x \]
And our table will include
\[ f(x) = x e^x \ ; \ \ \  F(x) = x e^x - e^x \]


\subsection*{inverse trig functions}

Inverse trig functions as the result of integration:

\[ \int \frac{1}{\sqrt{1-x^2}} \ dx = \sin^{-1} x \]
\[ \int \frac{1}{1 + x^2} \ dx = \tan^{-1} x \]
\[ \int \frac{1}{|x| \ \sqrt{1 + x^2}} \ dx = \sec^{-1} x \]

A moderately complicated one is 

\[ \int \frac{1}{\sqrt{1 + x^2}} \ dx =  \ln (\sqrt{1+x^2} + x ) \]

Check it by differentiating.  We obtain:
\[ \frac{d}{dx} \ \ln (\sqrt{1+x^2} + x ) \] 

\[ = (\frac{1}{\sqrt{1 + x^2} + x}) (\frac{2x}{2 \sqrt{1+x^2}} + 1) \]
\[ = (\frac{1}{\sqrt{1 + x^2} + x}) (\frac{x}{\sqrt{1+x^2}} + \frac{\sqrt{1+x^2}}{\sqrt{1+x^2}} ) \]
\[ = (\frac{1}{\sqrt{1 + x^2} + x}) (\frac{x + \sqrt{1+x^2}}{\sqrt{1+x^2}} ) \]
which does indeed simplify to
\[ = \frac{1}{\sqrt{1 + x^2}} \]

\subsection*{hyperbolic trig functions}

Without getting into the theory, the fundamental result about $\cosh x$ is
\[ \int \sinh x  \ dx = \cosh x \]
\[ \int \cosh x  \ dx = \sinh x \]

Notice the lack of a minus sign.  The reason for the name "hyperbolic" is this:

\[ \cosh^2 x - \sinh^2 x = 1 \]


\end{document}  