\documentclass[11pt, oneside]{article}   	% use "amsart" instead of "article" for AMSLaTeX format
\usepackage{geometry}                		% See geometry.pdf to learn the layout options. There are lots.
\geometry{letterpaper}                   		% ... or a4paper or a5paper or ... 
%\geometry{landscape}                		% Activate for for rotated page geometry
%\usepackage[parfill]{parskip}    		% Activate to begin paragraphs with an empty line rather than an indent
\usepackage{graphicx}				% Use pdf, png, jpg, or eps� with pdflatex; use eps in DVI mode
								% TeX will automatically convert eps --> pdf in pdflatex		
\usepackage{amssymb}
\usepackage{amsmath}
\usepackage{parskip}

\graphicspath{{/Users/telliott_admin/Dropbox/Tex/png/}}

\title{Trig Subs 2}
%\author{The Author}
\date{}							% Activate to display a given date or no date

\begin{document}
\maketitle
%\section{}
%\subsection{}
\Large

There are at least a couple more trig substitutions that are very useful for finding integrals.  One that we saw in looking at Euler's equation was:

\[ \int \frac{1}{\sqrt{1 + z^2}} \ i \ dz = i \ \ln \ (\sqrt{1+z^2} + z) \]
or generally
\[ \int \frac{1}{\sqrt{1 + x^2}} \ dx =  \ln \ (\sqrt{1+x^2} + x) \]

Of course, one way to see that this is correct is to differentiate the right-hand side:

\[ \frac{d}{dx} \ \ln \ (\sqrt{1+x^2} + x) \]
by the chain rule
\[ = \frac{1}{\sqrt{1+x^2} + x} \ (\frac{x}{\sqrt{1+x^2}} + 1) \]
We put the second term over a common denominator
\[ =  \frac{1}{\sqrt{1+x^2} + x} \ (\frac{x + \sqrt{1+x^2}}{\sqrt{1+x^2}}) \]
And now it's clear!
\[ = \frac{1}{\sqrt{1+x^2}} \]

The integral is normally written in a completely general way as

\[ \int \frac{1}{x^2 + a^2} \ dx \]

with $a$ a constant.  Our trig substitution is to draw a right-triangle with angle $y$ and side opposite $x$, and side adjacent $a$.  Then the hypotenuse is $\sqrt{x^2 + a^2}$.  We have:

\[ \frac{x}{a} = \tan y \]
\[ \frac{1}{a} \ dx = \sec^2 y \ dy \]

\[ \frac{a}{\sqrt{x^2+a^2}} = \cos y \]

So with the substitution the integral becomes
\[ \int \frac{1}{a} \ \cos y \ a \ \sec^2 y \ dy \]
which is just
\[ \int \sec y \ dy \]

We did this in the other section.  The trick is to multiply by 
\[ \int \sec y \ \frac{\sec y + \tan y}{\sec y + \tan y} \ dy \]
\[ \int  \frac{\sec^2 y + \sec y \tan y}{\sec y + \tan y} \ dy \]
This is just
\[ \int \frac{1}{u} \ du \]
so we have
\[ = \ln \ (\sec y + \tan y) \]

If we undo the substitution we obtain
\[ = \ln \ (\frac{\sqrt{x^2 + a^2}}{a} + \frac{x}{a} ) \]
and with $a=1$ this is just
\[ = \ln \ (\sqrt{x^2 + 1} + x ) \]

\subsection*{}

How about
\[ \int x^3 \sqrt{1-x^2} \ dx \]

This looks difficult, with that extra factor of $x^2$.  Try substituting
\[ x = \sin \theta \]
\[ dx = \cos \theta \ d \theta \]

We have
\[ = \int \sin^2 \theta \sqrt{1-\sin^2 \theta} \sin \theta \ d \theta \]
\[ = \int (1-\cos^2 \theta) \cos^2 \theta  \sin \theta \ d \theta \]
\[ = \int (-\cos^2 \theta + \cos^4 \theta) (-\sin \theta \ d \theta) \]
\[ = -\frac{1}{3} \cos^3 \theta + \frac{1}{5} \cos^5 \theta + C \]
\[ = -\frac{1}{3} (1-x^2)^{3/2} + \frac{1}{5} (1-x^2)^{5/2} + C \]

We can check this by differentiating, and then we'll understand what happened to the extra factor of $x^2$
\[ \frac{d}{dx} \ [ \  -\frac{1}{3} (1-x^2)^{3/2} + \frac{1}{5} (1-x^2)^{5/2} + C \ ] \]
\[ = x (1-x^2)^{1/2} - x (1-x^2)^{3/2} \]
\[ = x (1-x^2)^{1/2}  \ [ \ 1 - 1 + x^2  \ ] \]
and there it is!
\[ = x^3 (1-x^2)^{1/2}  \]
\[ = x^3 \sqrt{1-x^2}  \]

\subsection*{}

Here's another one that looks weird at first:

\[ \int \sqrt{x^2 + 6x} \ dx \]

Try completing the square

\[ = \int \sqrt{x^2 + 6x + 9 - 9} \ dx \]
\[ = \int \sqrt{(x + 3)^2 - 3^2} \ dx \]

Now substitute
\[ \frac{x+3}{3} = \sec t \]

$x+3$ is the hypotenuse, $3$ the side adjacent to angle $t$, and $\sqrt{(x + 3)^2 - 3^2}$ is the side opposite.  So

\[ \sqrt{(x + 3)^2 - 3^2} = 3 \tan t \]

and for $dx$:

\[ \frac{x+3}{3} = \sec t \]
\[ \frac{1}{3} \ dx = \sec t \tan t \ dt  \]

So our integral is 

\[ = \int 3 \tan t \ 3 \sec t \tan t \ dt \]

Recall that $\tan^2 t + 1 = \sec^2 t$

\[ = 9 \int \sec^3 t - \sec t \ dt \]

We had a trick for $\sec t$ which gives 
\[ \int \sec t \ dt = \ln | \sec t + \tan t | + C \]

(easily checked by differentiating).  The other term is solved by integration by parts.  I'll just give a sketch here:

\[ \int sec^3 t \ dt = \sec t \tan t - \int \sec^3 t \ dt + \int \sec t \ dt \]
\[ \int sec^3 t \ dt = \frac{1}{2} \ [ \ \sec t \tan t + \int \sec t \ dt \ ] \]

combined with what was above, we end up subtracting (one-half) $\int \sec t \ dt$

\[ = (9) \  \frac{1}{2} \ [ \ \sec t \tan t - \ln | \sec t + \tan t | \ ]  + C \]

I'll leave it to you to substitute back for $x$.


\end{document}  