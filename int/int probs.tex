\documentclass[11pt, oneside]{article}   	% use "amsart" instead of "article" for AMSLaTeX format
\usepackage{geometry}                		% See geometry.pdf to learn the layout options. There are lots.
\geometry{letterpaper}                   		% ... or a4paper or a5paper or ... 
%\geometry{landscape}                		% Activate for for rotated page geometry
%\usepackage[parfill]{parskip}    		% Activate to begin paragraphs with an empty line rather than an indent
\usepackage{graphicx}				% Use pdf, png, jpg, or eps� with pdflatex; use eps in DVI mode
								% TeX will automatically convert eps --> pdf in pdflatex		
\usepackage{amssymb}
\usepackage{amsmath}
\usepackage{parskip}

\title{Integration problems}
%\author{The Author}
%\section{}
% \subsection*{R code}
\date{}							% Activate to display a given date or no date

\graphicspath{{/Users/telliott_admin/Dropbox/Tex/png/}}

\begin{document}
\maketitle
\Large
%\noindent

Definite integrals

\[ \int_{0}^{\pi/2} \cos x \ dx \]
\[ \int_{0}^{\pi/2} \cos \theta \ d\theta \]
\[ \int_{-\pi/2}^{\pi/2} \sin x \ dx \]
\[ \int_0^1 x^2 \ dx \]
\[ \int_0^1 \sqrt{x} \ dx \]

\[ \int_1^{3} \frac{1}{x} \ dx \]
\[ \int_1^2 3x^2 + 2x + 1 \ dx \]

\subsection*{}

\[ \int_{0}^{8} x^{2/3} \ dx \]
\[ \int_{0}^{\pi/4}  \cos 2t \ dt \]
\[ \int_{1}^3 \frac{1}{x^2} \ dx \]

\subsection*{}

\[ \int_0^3 \frac{2t}{1+t^2} \ dt \ ; \ \ \text{hint:  watch the bounds} \]
\[ = \int_{u=1}^{u=10} \frac{1}{u} \ du \]
\[ = \ln u \ \bigg |_{u=1}^{u=10} = \ln 10 - \ln 1 = \ln 10 \]

\subsection*{}

\[ \int_0^1 (3x - 2)^3 \ dx \]
\[ \int_0^1 xe^{-x^2} \ dx \]
\[ \int_{-\pi/4}^{\pi/4} \cos 2x \ dx \]
\[ \int_{-1}^1 xe^x \ dx \]
\[ \int_0^{1/2} \frac{1}{\sqrt{1-x^2}} \ dx \]
\[ \int_0^{e-1} \ln (x + 1) \ dx ; \ \ \text{hint: what is } \frac{d}{dx} x \ln x \ \text{?}\]
\[ \int_{\pi/3}^{\pi/2} \tan \frac{\theta}{2} \ \sec^2 \frac{\theta}{2}\ d \theta \]
\[ \int_{\pi/2}^{x} \cos t \ dt \]
\[ \int_0^{\ln 2} e^{2x} \ dx \]
\[ \int_0^2 (x^3 + k) \ dx = 10 \ ; \ \ \text{find } k \]
\[ \int_0^{\ln 2} e^{2x} \ dx \]
\[ \int_1^e \frac{\ln t}{t} \ dt \]
\[ \int_0^1 x e^{x^2 + 1} \ dx \]

Indefinite integrals

\[ \int \tan x \ dx \]
\[ \int \ln x \ dx \]
\[ \int \sec^2 x \ dx \]
\[ \int \csc^2 \theta \ d \theta \]
\[ \int \tan \theta \sec \theta \ d\theta \]

\subsection*{}

\[ \int (\sqrt{x} + \frac{1}{x^3}) \ dx \]
\[ \int \frac{3x^2 + x - 1}{x^2} \ dx \]
\[ \int \frac{1}{u-3} \ du \]
\[ \int \cos^2(2x) \ \sin 2x \ dx \]
\[ \int \frac{x}{\sqrt{3-4x^2}} \ dx \]
\[ \int \frac{1}{\sqrt{9-x^2}} \ dy \]
\[ \int \frac{x}{(2-x^2)^3} \ dx \]
\[ \int \frac{e^x}{1-2e^x} \ dx \]
\[ \int e^{x + e^x} \ dx \]

\subsection*{}

\[ \int (x^3 - \sin 2x) \ dx \]
\[ \int \frac{e^{3x}}{e^x} \ dx \]
\[ \int \frac{z}{1-4z^2} \ dz \]
\[ \int \frac{5}{1 + x^2} \ dx \]
\[ \int \frac{\cos x}{\sin^2 x} \ dx \]
\[ \int \tan^4 t \ \sec^2 t \ dt \]
\[ \int e^x \cos (e^x) \ dx \]
\[ \int \frac{e^x - e^{-x}}{e^x + e^{-x}} \ dx \]
\[ \int \frac{x+1}{x^2 + 1} \ dx \]
\[ \int \frac{x}{x+a} \ dx \ ; \ \ \text{hint:  } =  \int \frac{x + a - a}{x+a} \ dx \]
\[ \int a^u \ du  \ ; \ \ a = const \]

\subsection*{}

\[ \int e^{4- \ln x} \ dx \]
\[ \int x \ \sqrt{x+2} \ dx \ ; \ \ \text{hint:  } u = x+2 \]
\[ \int \frac{x}{\sqrt{x+3}} \ dx \ ; \ \ \text{hint:  } u = x+3 \]
\[ \int \frac{1+x}{\sqrt x} \ dx \]
\[ \int \frac{1}{x^2 + 2x + 5} \ dx \]
\[ \int \frac{x^2 + 3}{x-1} \ dx \ : \ \ \text{hint:  make top divisible by } x-1 \]
\[ \int \frac{\ln x}{3x} \ dx \]
\[ \int \frac{e^x}{e^x + 1} \ dx \]
\[ \int \frac{1+x}{\sqrt x} \ dx \]
\[ \int \sin \theta \cos \theta \ d \theta \]
for the last, give both versions of the answer and show they are equal

\subsection*{}

\[ \int (\sin x + \cos x)^2 \ dx \]
\[ \int (1 + \tan x)^2 \ dx \]
\[ \int \frac{\cos^2 x}{1 + \sin x} \ dx \]
\[ \int \frac{\sin x}{1 + \sin x} \ dx \]
\[ \int \sin^3 \ dx \]
\[ \int \sec^2 x \ \sqrt{5 + \tan x} \ dx \]
\[ \int \cos x \ e^{1 + \sin x} \ dx \]
\[ \int e^x \cos (e^x) \ dx \]
\[ \int x \sin x \ dx \]
\[ \int e^x \sin x \ dx \]
\[ \int \frac{\sin x + \cos x}{e^{-x} + \sin x} \ dx  \ ; \ \ \text{hint:  multiply by }  e^x/e^x \]
\[ \int \frac{2^{\ln x}}{x} \ dx \]
\[ \int \frac{1}{x \ln x } \ dx \]
\[ \int \frac{\ln \sqrt{x}}{x} \ dx \]

\subsection*{}

For absolute value problems, recall that
 \begin{displaymath}
   |x| = \left\{
     \begin{array}{lr}
       \ \ x & : x \ge 0 \\
       -x & : x < 0
     \end{array}
   \right.
\end{displaymath} 

The method is to find the place where the expression inside the absolute value symbols is equal to zero, then integrate piecewise, substituting as shown above.

\[ \int_0^2 |t-1| \ dt \]

Since $t-1 = 0$ when $t=1$ this is

\[ \int_0^1 -(t-1) \ dt + \int_1^2 (t-1) \ dt \]
\[ = (-\frac{1}{2} t^2 + t) \  \bigg |_0^1 \ + (\frac{1}{2} t^2 - t) \ \bigg |_1^2 \]
\[ = (-\frac{1}{2}  + 1  - 0 + 0) + (2 - 2 - \frac{1}{2} + 1) = 1 \]
Tricky to evaluate.

\subsection*{FTC}

There is a perverse desire to make sure you understand the FTC (part 1).

If $F(x)$ is "nice" and
\[ F(x) = \int_a^x f(t) \ dt \]
then..
\[ F'(x) = \frac{d}{dx} \ F(x) = \frac{d}{dx} \  \int_a^x f(t) \ dt = f(x) \]

Problems:  for each $G(x)$ below, find $G'(x)$

\[ G(x) = \int_1^x 2t \ dt \]
\[ G(x) = \int_0^x (2t^2 + \sqrt{t}) \ dt \]
\[ G(x) = \int_0^x \tan t \ dt \]
\[ G(x) = \int_{x^2}^x \frac{t^2}{1+t^2} \ dt \]

The last problem needs first to be manipulated into a (sum of) integrals between a constant ($0$) on the lower bound and $x$ above, and then the one with $x^2$ must take account of the fact that if $t=-x^2$ then $dt = -2x \ dx$.

\[ G(x) = - \int_0^{-x^2} \frac{t^2}{1+t^2} \ dt +   \int_0^x \frac{t^2}{1+t^2} \ dt \]
\[ G'(x) = - \frac{x^4}{1 + x^4} \ (-2x) + \frac{x^2}{1+x^2} \]

\subsection*{hard one}

Here is a problem involving the actual integral we had above. I didn't know how to solve it completely, but I found the answer on the web and can work backward and see that it's correct.  Call it a challenge.  It looks simple enough:

\[ \int \frac{\sqrt{x}}{1-x} \ dx \]
substitute
\[ u = \sqrt{x}, \ \ u^2 = x, \ \ 2 u \ du = dx \]
we obtain
\[ \int \frac{u}{1-u^2} \ 2 u \ du \]
\[ 2 \int \frac{u^2}{1-u^2} \ du \]

If this were $x$ in the numerator rather than $x^2$, it would be simple.  Still, it looks like it ought to be easy, somehow.  The answer is here:

(http://integrals.wolfram.com/index.jsp)

Let's change to $x$:

\[ \int \frac{x^2}{1-x^2} \ dx \]

The first part of the answer is a useful trick for many problems.  If the numerator is the same as the denominator, within a constant, then:

\[ = - \int \frac{1-x^2 - 1}{1-x^2} \ dx \]
\[ = -\int 1 \ dx - \int \frac{1}{1-x^2} \ dx \]

Now the real trick is that the second part can be re-worked because it is a difference of squares 

\[ (1-x)(1+x) = 1-x^2 \]
\[ \int \frac{1}{1-x^2} \ dx = \frac{1}{2} \int ( \frac{1}{1+x} + \frac{1}{1-x} ) \ dx \]

If we put the two terms on the right over the common denominator $1-x^2$, then for the numerator we have $1-x + 1 + x = 2$.  !!  So the whole integral is

\[ \int \frac{x^2}{1-x^2} \ dx \]
\[ = -x - \frac{1}{2} \ [ \ (\ln (1+x) - \ln(1-x) ) \ ] \ + C \]
\[ = -x - \frac{1}{2} \ \ln \frac{(1+x)}{(1-x)} + C \]

I'll leave it to you to work out the answer to the original problem with $\sqrt{x}$.

\subsection*{another hard one}

\[ \int \frac{\sqrt{x^2 + 1}}{x} \ dx \]

Substitution:  let $x=\tan t$.  So opp = $x$, adj = $1$, hyp = $\sqrt{1 + x^2}$.

\[ x = \tan t \]
\[ dx = \sec^2 t \ dt \]
\[ \sqrt{1 + x^2} = \sec t \]
So the integral is

\[ \int \frac{\sec t}{\tan t} \ \sec^2 t \ dt \]
\[ = \int \frac{\sec t}{\tan t} (1 + \tan^2 t) \ dt \]

The first term is

\[ \int \frac{1}{\sin t} \ dt \]
and the second is

\[ \int \sec t \tan t \ dt \]
\[ = \int \frac{\sin t}{\cos^2 t} \ dt \]

The second part is easy ($1 / \cos t$).  But the first requires more work.  Let 

\[ u = \cos t \]
\[ du = - \sin t \ dt \]

We rewrite the integral as

\[ \int \frac{\sin t}{\sin^2 t} \ dt \]

\[ = - \int \frac{1}{1 - u^2} \ du \]
\[ = - \frac{1}{2} \int \frac{1}{1+u} + \frac{1}{1-u} \ du \]
\[ = - \frac{1}{2} ( \ln (1+u) - \ln (1-u) ) \]

So, in terms of $t$ we have (combining)

\[ \frac{1}{\cos t } - \frac{1}{2} ( \ln (1+\cos t) - \ln (1-\cos t) ) \] 

In order to substitute back to $x$, we recall that

\[ \frac{1}{\sqrt{1 + x^2}} = \cos t \]

and I think we'll just leave it right there.  Well, in the original problem we had a definite integral with limits $\sqrt{15}$ and $\sqrt{3}$, so that $\cos t = 1/4$ at the high end and $\cos t = 1/2$ at the low end which makes it considerably easier to evaluate.

\[ = 4 - \frac{1}{2} (\ln 5/4 - \ln 1/2) - 2 + \frac{1}{2} (\ln 3/2 - \ln 1/2) \]
\[ = 2 - \frac{1}{2}  ( \ln 5/2 +  \ln 3 ) \]



\end{document}  