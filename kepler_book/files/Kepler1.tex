\documentclass[11pt, oneside]{article} 
\usepackage{geometry}
\geometry{letterpaper} 
\usepackage{graphicx}
	
\usepackage{amssymb}
\usepackage{amsmath}
\usepackage{parskip}
\usepackage{color}
\usepackage{hyperref}

\graphicspath{{/Users/telliott_admin/Dropbox/Tex/png/}}
% \begin{center} \includegraphics [scale=0.4] {gauss3.png} \end{center}

\title{Kepler 1 (Varberg)}
\date{}

\begin{document}
\maketitle
\Large

K2 says that orbits "sweep out equal areas in equal times" or equivalently, that $\ddot{A} = d^2A/dt^2 = 0$.  So we start from Newton's force directed toward the sun, the centripetal force, and we will show that the motion stays in a plane, and also implies K2.

Again, this is Feynman's argument (and notation)
\[ \frac{d}{dt} ( \mathbf{r} \times \mathbf{v}) = \frac{d}{dt} ( \mathbf{r} \times \dot{\mathbf{r}}) = 0 \]
This is zero because you get two terms from the derivative of the cross-product:  one is $\dot{\mathbf{r}} \times \dot{\mathbf{r}} = 0$ , and the second one is $\mathbf{r} \times \ddot{\mathbf{r}}$, which is zero because these two vectors point in opposite directions by the centripetal force postulate.
Therefore,  $\mathbf{r} \times \dot{\mathbf{r}}$ is constant.  We will say that
\[ \mathbf{r} \times \dot{\mathbf{r}} = \mathbf{h} \]
\[ |\mathbf{h}| = h = 2 \ \frac{dA}{dt} \]
If $\mathbf{h} = 0$, there is no force, and just straight-line motion.  But for  $\mathbf{h} \ne 0$, then $\mathbf{r}$ and $\dot{\mathbf{r}}$ are in a plane that doesn't change with time, and $\mathbf{h}$ is the normal vector of that plane.
\[ h = | \mathbf{r} \times \dot{\mathbf{r}} | = |\mathbf{r} \times \frac{d\mathbf{r}}{dt} | \]
by the discussion about area previously (Varberg's Example $14.5$)
\[ h =  r^2 \ \frac{d \theta}{dt} = 2 \ \frac{dA}{dt} \]
This is the statement of K2.  $dA/dt$ is constant, equal to $h/2$.

At this point, Varberg reverse the argument and show that planar motion and K2 imply a centripetal force.  But this is just Feynman's dots, which we already went through.
\subsection*{Kepler's First Law K1}
Now, we make an additional hypothesis due to Newton, which is that the acceleration is proportional to the inverse square of the distance from the sun (origin), and pointed toward it.
\[ \mathbf{a} = \ddot{\mathbf{r}} = - \frac{GM}{r^2} \mathbf{u_r} \]
where (as before) $\mathbf{u_r}$ is the unit vector in the $\mathbf{r}$ direction (i.e. equal to $\mathbf{r}/|\mathbf{r}|$), and $GM$ is a constant.

The first of three main steps in the proof is to take the cross-product with $\hat{\mathbf{k}}$ (as the text says, "this allows us to introduce the area information in vectorial form")
\[ \ddot{\mathbf{r}} \times \hat{\mathbf{k}} = - \frac{GM}{r^2} \ \mathbf{u_r} \times \hat{\mathbf{k}} = \frac{GM}{r^2} \ \mathbf{u_\theta} \]
(recall that we "go to the left" for $\mathbf{u_\theta}$).  

It should not be surprising that the cross-product $\mathbf{u_r} \times \hat{\mathbf{k}} $ brings us back to $- \mathbf{u_\theta}$, since we defined
\[ \mathbf{u_r} \times \mathbf{u_{\theta}} = \hat{\mathbf{k}} \]
\[  \hat{\mathbf{k}} \times \mathbf{u_r} = \mathbf{u_{\theta}} \]
and thus
\[  \mathbf{u_r} \times\hat{\mathbf{k}} = - \mathbf{u_{\theta}} \]

From our discussion of the unit vectors and parametrization,
\[ \frac{d}{dt} \ \mathbf{u_r} = \dot{\mathbf{u}}_\mathbf{r} = \frac{d\theta}{dt} \ \mathbf{u_\theta} = \omega \mathbf{u_\theta} \]
\[ \mathbf{u_\theta} = \frac{\dot{\mathbf{u}}_{\mathbf{r}}}{\omega} \]
and from K2
\[ \frac{d \theta}{dt} = \omega = \frac{h}{r^2} \]
Hence
\[  \mathbf{u_\theta} =  \frac{\dot{\mathbf{u}}_{\mathbf{r}}}{\omega} = \frac{\dot{\mathbf{u}}_{\mathbf{r}}}{h/r^2} \]

So the cross-product which we had as
\[ \ddot{\mathbf{r}} \times \hat{\mathbf{k}} = \frac{GM}{r^2} \  \mathbf{u_\theta} \]
is equal to
\[ = \frac{GM}{r^2} \  \frac{\dot{\mathbf{u}}_{\mathbf{r}}}{h/r^2} \]
\[ = \frac{GM}{h} \ \dot{\mathbf{u}}_{\mathbf{r}}  \]
This is really the key step in the whole adventure.

The second clever thing is to integrate with respect to time
\[ \int \ddot{\mathbf{r}} \times \hat{\mathbf{k}} = \int \frac{GM}{h} \ \dot{\mathbf{u}}_{\mathbf{r}}  \]
(remember that $GM$, $h$ and $\hat{\mathbf{k}}$ are all constant)
\[ \dot{\mathbf{r}} \times \hat{\mathbf{k}} = \frac{GM}{h} \ ( \mathbf{u_r} + \mathbf{E}) \]
where $ \mathbf{E}$ is a constant (vector) of integration.

The third step is to realize that we can simplify a lot by forming the dot product of both sides with $\mathbf{r}$
\[ \mathbf{r} \cdot ( \dot{\mathbf{r}} \times \hat{\mathbf{k}}) = \frac{GM}{h} \ \mathbf{r} \cdot ( \mathbf{u_r} + \mathbf{E}) \]
using a vector identity, the left-hand side is
\[ \mathbf{r} \cdot ( \dot{\mathbf{r}} \times \hat{\mathbf{k}}) = (\mathbf{r} \times \dot{\mathbf{r}}) \cdot \hat{\mathbf{k}} \]
but 
\[ \mathbf{r} \times \dot{\mathbf{r}} = \mathbf{h} =  h \ \hat{\mathbf{k}} \]
so we have
\[ h \ \hat{\mathbf{k}} \cdot \hat{\mathbf{k}} = h \]

Bringing back the right-hand side:
\[ \mathbf{r} \cdot (\ddot{\mathbf{r}} \times \hat{\mathbf{k}}) = h = \frac{GM}{h} \ \mathbf{r} \cdot ( \mathbf{u_r} + \mathbf{E}) \]
\[ \frac{h^2}{GM} =  \mathbf{r} \cdot ( \mathbf{u_r}+ \mathbf{E}) \]
Recall that $ \mathbf{u_r}$ is the unit vector in the same direction as $\mathbf{r}$ so that $\mathbf{r} \cdot  \mathbf{u_r} = r$.

We can take $\mathbf{E}$ to be in the direction of $\mathbf{r}$ at time-zero so $\mathbf{r} \cdot \mathbf{E}$ is equal to $r$ times $e$ times the cosine of the angle between them at some later time.  (Since $\mathbf{E}$ is a constant vector of integration, its magnitude $e$ can be anything).

This becomes 
\[ r(1 + e \cos \theta) = \frac{h^2}{GM} \]
which is an equation in polar coordinates.

This is a family of curves which are conic sections.  If $e < 1$ it's an ellipse.
\begin{center} \includegraphics [scale=0.6] {quick_ellipse.png} \end{center}
The curve in the figure is an ellipse with the formula
\[ r(1 + 0.8 \cos \theta) = 1 \]
$e$ is the eccentricity of the ellipse
\[ e^2 +  \frac{b^2}{a^2} = 1 \]
In the figure 
\[ e^2 = 0.8^2 = 0.64 \]
\[\frac{b^2}{a^2} = 1 - 0.64 = 0.36 \]
\[\frac{b}{a} = \sqrt{0.36} = 0.6 \]

\end{document}  